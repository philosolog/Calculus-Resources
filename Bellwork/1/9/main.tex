\documentclass[12pt]{beamer}
\usetheme{Warsaw}
\usepackage[utf8]{inputenc}
\usepackage{amsmath}
\usepackage{amsfonts}
\usepackage{amssymb}
\usepackage{graphicx}
\usepackage[font=Times,timeinterval=1,timeduration=2.0,timedeath=0,fillcolorwarningsecond=white!60!yellow,timewarningfirst=50,timewarningsecond=80,resetatpages=2]{tdclock}
\usepackage{tabularx}
\usepackage{array}
\usepackage{multicol}
\usepackage{longtable}
\usepackage{xcolor}
\usepackage{textcomp, gensymb}
\usepackage{pgfplots}
\usepackage[makeroom]{cancel}

\graphicspath{ {./references/} }
\pgfplotsset{
	soldot/.style={color=black,only marks,mark=*},
	holdot/.style={color=black,fill=white,only marks,mark=*},
	compat=1.12
}
\newcolumntype{Y}{>{\centering\arraybackslash}X}
\makeatletter
\def\@listii{\leftmargin\leftmarginii
			  \topsep    2ex
			  \parsep    0\p@   \@plus\p@
			  \itemsep   \parsep}
\makeatother
\newcommand\at[2]{\left.#1\right|_{#2}}

\begin{document}
\begin{frame}
	\frametitle{Bellwork 1/9}
	\initclock

	\vfill
	\vfill
	\large
	Let $\int_{0}^{3}[4f(x)+2]dx=18$ and $\int_{3}^{6}f(x)dx=-1$.
	\vfill
	Find:
	\Large
	\begin{enumerate}\itemsep2ex
		\item $\int_{0}^{3}f(x)dx$
		\item $\int_{6}^{0}f(x)dx$
	\end{enumerate}
	\vfill
	\vfill
	\vfill
	\vfill

	\small
	\crono
	\resetcrono{\beamerbutton{reset}}
\end{frame}
\begin{frame}
	\frametitle{Bellwork 1/9 - Solution, Part 1}

	\large
	\begin{align*}
		\int_{0}^{3}[4f(x)+2]dx               & = 18 \\
		4\int_{0}^{3}f(x)dx + 2\int_{0}^{3}dx & = 18 \\
		4\int_{0}^{3}f(x)dx + 6               & = 18
	\end{align*}
	\begin{center}
		$\implies \boxed{\int_{0}^{3}f(x)dx = 3}$
	\end{center}
\end{frame}
\begin{frame}
	\frametitle{Bellwork 1/9 - Solution, Part 2}

	\begin{align*}
		\int_{6}^{0}f(x)dx &= -\int_{0}^{6}f(x)dx \\
		&= -\left[\int_{0}^{3}f(x)dx + \int_{3}^{6}f(x)dx\right]
	\end{align*}
	\vfill
	\hrule
	\vfill
	\[\int_{0}^{3}f(x)dx=3\text{; }\int_{3}^{6}f(x)dx=-1\]
	\[\implies \boxed{\int_{6}^{0}f(x)dx = -3 + 1 = -2}\]
\end{frame}
\begin{frame}
	\frametitle{Exercise 1}

	\vfill
	\vfill
	\Large
	\[f(x)=\int_{1}^{x}\ln(1+t^2)dt\]
	\vfill
	\begin{center}
		Find $f'(x)$ using part 1 of the FTC.
	\end{center}
	\vfill
	\vfill
\end{frame}
\begin{frame}
	\frametitle{Exercise 1 - Solution}

	\vfill
	\Large
	\begin{align*}
		f(x)           & = \int_{1}^{x}\ln(1+t^2)dt                               \\
		\implies f'(x) & = \frac{\mathrm{d}}{\mathrm{d}x}\int_{1}^{x}\ln(1+t^2)dt
	\end{align*}
	\vfill
	\[\text{FTC 1}\implies \boxed{f'(x) = \ln(1+x^2)}\]
\end{frame}
\begin{frame}
	\frametitle{Exercise 2}

	\vfill
	\vfill
	\Large
	\[h(t)=\int_{2}^{t}\frac{s}{s^4+1}ds\]
	\vfill
	\begin{center}
		Find $h'(t)$ using part 1 of the FTC.
	\end{center}
	\vfill
	\vfill
\end{frame}
\begin{frame}
	\frametitle{Exercise 2 - Solution}

	\vfill
	\Large
	\begin{align*}
		h(t)           & = \int_{2}^{t}\frac{s}{s^4+1}ds                               \\
		\implies h'(t) & = \frac{\mathrm{d}}{\mathrm{d}t}\int_{2}^{t}\frac{s}{s^4+1}ds
	\end{align*}
	\vfill
	\[\text{FTC 1}\implies \boxed{h'(t) = \frac{t}{t^4+1}}\]
\end{frame}
\end{document}