\documentclass[12pt]{beamer}
\usetheme{Warsaw}
\usepackage[utf8]{inputenc}
\usepackage{amsmath}
\usepackage{amsfonts}
\usepackage{amssymb}
\usepackage{graphicx}
\usepackage[font=Times,timeinterval=1,timeduration=2.0,timedeath=0,fillcolorwarningsecond=white!60!yellow,timewarningfirst=50,timewarningsecond=80,resetatpages=2]{tdclock}
\usepackage{tabularx}
\usepackage{array}
\usepackage{multicol}
\usepackage{longtable}
\usepackage{xcolor}
\usepackage{textcomp, gensymb}
\usepackage{pgfplots}
\usepackage[makeroom]{cancel}

\graphicspath{ {./references/} }
\pgfplotsset{
	soldot/.style={color=black,only marks,mark=*},
	holdot/.style={color=black,fill=white,only marks,mark=*},
	compat=1.12
}
\newcolumntype{Y}{>{\centering\arraybackslash}X}
\makeatletter
\def\@listii{\leftmargin\leftmarginii
			  \topsep    2ex
			  \parsep    0\p@   \@plus\p@
			  \itemsep   \parsep}
\makeatother
\newcommand\at[2]{\left.#1\right|_{#2}}
\setlength{\abovedisplayskip}{-15pt}
\setlength{\belowdisplayskip}{0pt}
\setlength{\abovedisplayshortskip}{0pt}
\setlength{\belowdisplayshortskip}{0pt}

\begin{document}
\begin{frame}
	\frametitle{Bellwork 1/12}
	\initclock

	\vfill
	\vfill
	\vfill
	\Large
	Find the general indefinite integral:
	\vfill
	\[\int (2+\tan^2\theta)d\theta\]
	\vfill
	\large
	\[\text{Hint: }1+\tan^2\theta=\sec^2\theta\]
	\vfill
	\vfill
	\vfill
	\vfill

	\small
	\crono
	\resetcrono{\beamerbutton{reset}}
\end{frame}
\begin{frame}
	\frametitle{Bellwork 1/12 - Solution}

	\Large
	\begin{align*}
		\int (2+\tan^2\theta)d\theta&=\int (1 + \sec^2\theta)d\theta \\
		&=\int d\theta + \int \sec^2\theta d\theta\\
		&=\boxed{\theta + \tan\theta + C}
	\end{align*}
\end{frame}
\begin{frame}
	\frametitle{Exercise 1}

	\large
	The velocity function $\left(\text{in }\frac{\text{m}}{\text{s}}\right)$ is given for a particle moving along a line. \\
	\[v(t)=6t-5\text{; }0\leq t\leq 5\]
	\vfill
	For the given $t$ interval, find the particle's:
	\begin{enumerate}
		\item Displacement
		\item Distance Traveled
	\end{enumerate}
\end{frame}
\begin{frame}
	\frametitle{Exercise 1 - Solution: Displacement}

	\large
	\begin{align*}
		\int_{0}^{5}v(t)dt &= s(5)-s(0)=\Delta s=\text{displacement}\\
		\implies \Delta s &= \int_{0}^{5}6t-5dt\\
		&= 6\int_{0}^{5}tdt-5\int_{0}^{5}dt\\
		&= 3[t^2]_0^5-5[t]_0^5\\
		&= \boxed{50\text{ meters}}
	\end{align*}
\end{frame}
\begin{frame}
	\frametitle{Exercise 1 - Solution: Distance Traveled}

	Find where $v(t)$ changes sign:
	\begin{align*}
		v(t)=6t-5&=0\\
		\implies t&=\frac{5}{6}
	\end{align*}
	Create the integral:
	\begin{align*}
		\int_{0}^{5}|v(t)|dt&=\left|\int_{0}^{\frac{5}{6}}6t-5dt\right|+\left|\int_{\frac{5}{6}}^{5}6t-5dt\right|\\
		&= \left|[3t^2-5t]_0^\frac{5}{6}\right|+\left|[3t^2-5t]_\frac{5}{6}^5\right|\\
		&= \boxed{\frac{325}{6}=54.1\overline{6}\text{ meters}}
	\end{align*}
\end{frame}
\begin{frame}
	\frametitle{Exercise 2}

	\large
	The velocity function $\left(\text{in }\frac{\text{m}}{\text{s}}\right)$ is given for a particle moving along a line. \\
	\[v(t)=6t^2+2t-4\text{; }0\leq t\leq 4\]
	\vfill
	For the given $t$ interval, find the particle's:
	\begin{enumerate}
		\item Displacement
		\item Distance Traveled
	\end{enumerate}
\end{frame}
\begin{frame}
	\frametitle{Exercise 2 - Solution: Displacement}

	\large
	\begin{align*}
		\int_{0}^{4}v(t)dt &= s(4)-s(0)=\Delta s=\text{displacement}\\
		\implies \Delta s &= \int_{0}^{4}6t^2+2t-4dt\\
		&= 6\int_{0}^{4}t^2dt+2\int_{0}^{4}tdt-4\int_{0}^{4}dt\\
		&= 2[t^3]_0^4+[t^2]_0^4-4[t]_0^4\\
		&= \boxed{128\text{ meters}}
	\end{align*}
\end{frame}
\begin{frame}
	\frametitle{Exercise 2 - Solution: Distance Traveled, Part 1}

	Find where $v(t)$ changes sign:
	\begin{align*}
		v(t)=6t^2+2t-4&=0\\
		\implies t&=\cancel{-1}\text{, }\frac{2}{3}\text{ ($t\in [0,4]$)}
	\end{align*}
	Create the integral:
	\begin{align*}
		\int_{0}^{5}|v(t)|dt&=\left|\int_{0}^{\frac{2}{3}}6t^2+2t-4dt\right|+\left|\int_{\frac{2}{3}}^{4}6t^2+2t-4dt\right|\\
		&= \left|[2t^3+t^2-4t]_0^\frac{2}{3}\right|+\left|[2t^3+t^2-4t]_\frac{2}{3}^4\right|\\
		&= \boxed{\frac{3544}{27}\approx131.259\text{ meters}}
	\end{align*}
\end{frame}
\end{document}