\documentclass[12pt]{beamer}
\usetheme{Warsaw}
\usepackage[utf8]{inputenc}
\usepackage{amsmath}
\usepackage{amsfonts}
\usepackage{amssymb}
\usepackage{graphicx}
\usepackage[font=Times,timeinterval=1,timeduration=2.0,timedeath=0,fillcolorwarningsecond=white!60!yellow,timewarningfirst=50,timewarningsecond=80,resetatpages=2]{tdclock}
\usepackage{tabularx}
\usepackage{array}
\usepackage{multicol}
\usepackage{longtable}
\usepackage{xcolor}
\usepackage{gensymb}
\usepackage{pgfplots}
\usepackage{gensymb}

\newcolumntype{Y}{>{\centering\arraybackslash}X}
\begin{document}
\begin{frame}
	\frametitle{Bellwork 8/23 (5 Minutes)}
	\initclock
	\textbf{[Linear Models]}\vspace{.2cm}\\
	\vspace*{\fill}
	On an arbitrarily chosen Planet X, the average global temperature $(Q_{\text{avg}})$ is currently increasing at a rate of $0.156\degree$C per decade. Right now, the planet's average global temperature is at $21.4\degree$C.
	\vspace*{\fill}
	\begin{enumerate}
		\item Write an equation that models the average global temperature of Planet X in $t$ decades from now.
		      \vspace*{\fill}
		\item Using your linear model, approximate the amount of time it will take for the planet to exceed a $Q_{\text{avg}}$ of $25\degree$C.
	\end{enumerate}
	\vspace*{\fill}
	\vspace*{\fill}
	\crono
	\resetcrono{\beamerbutton{reset}}
\end{frame}
\end{document}