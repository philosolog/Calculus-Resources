\documentclass[12pt]{beamer}
\usetheme{Warsaw}
\usepackage[utf8]{inputenc}
\usepackage{amsmath}
\usepackage{amsfonts}
\usepackage{amssymb}
\usepackage{graphicx}
\usepackage[font=Times,timeinterval=1,timeduration=2.0,timedeath=0,fillcolorwarningsecond=white!60!yellow,timewarningfirst=50,timewarningsecond=80,resetatpages=2]{tdclock}
\usepackage{tabularx}
\usepackage{array}
\usepackage{multicol}
\usepackage{longtable}
\usepackage{xcolor}
\usepackage{gensymb}
\usepackage{pgfplots}
\usepackage[makeroom]{cancel}

\graphicspath{ {./references/} }
\pgfplotsset{
	soldot/.style={color=black,only marks,mark=*},
	holdot/.style={color=black,fill=white,only marks,mark=*},
	compat=1.12
}
\newcolumntype{Y}{>{\centering\arraybackslash}X}
\makeatletter
\def\@listii{\leftmargin\leftmarginii
			  \topsep    2ex
			  \parsep    0\p@   \@plus\p@
			  \itemsep   \parsep}
\makeatother

\begin{document}
\begin{frame}
	\frametitle{Bellwork 9/22}

	\large
	\initclock
	\begin{center}
		Let $f(x) =
			\begin{cases}
				2x - 3      & \text{ if } x < 1    \\
				\cos(\pi x) & \text{ if } x \geq 1 \\
			\end{cases}$ \\
	\end{center}
	\vfill
	Does the Intermediate Value Theorem guarantee a solution to $f(x) = -0.25$ in the interval $(0\text{, } 2)$?\par
	\vfill
	Why or why not?\par
	\vfill
	\vfill
	\crono
	\resetcrono{\beamerbutton{reset}}
\end{frame}
\begin{frame}
	\frametitle{Bellwork 9/22 - Solution, Part 1}

	\boxed{\text{Yes!..}}%, the IVT does guarantee a value $x$ in $(0\text{, } 2)$ such that $f(x)=-0.25$.\par
	\vfill
	In order to use the IVT here, $f$ must be continuous on $[0\text{, } 2]$.\par
	\vfill
	Since $2x-3$ and $\cos(\pi x)$ are continuous for all $x$, we just need to test for the continuity of $f$ at $x=1$:\par
	\[\displaystyle\lim_{x\to1^{-}}f(x)\stackrel{?}{=}\displaystyle\lim_{x\to1^{+}}f(x)\stackrel{?}{=}f(1)\]
	\[2(1)-3=\cos[\pi(1)]\implies -1=-1\]
	\[\text{$\therefore$ $f$ is continuous at $x=1$}\]
\end{frame}
\begin{frame}
	\frametitle{Bellwork 9/22 - Solution, Part 2}

	\vfill
	\vfill
	\vfill
	Since $f$ is continuous on $[0\text{, } 2]$, we now check if $-0.25$ is between $f(0)$ and $f(2)$:
	\[\text{$f(0)=-3$ and $f(2)=1$}\]
	\vfill
	Because $f(0) < -0.25 < f(2)$, the IVT guarantees a value $x$ in $(0\text{, } 2)$ such that $f(x)=-0.25$.
	\vfill
	\vfill
	\vfill
\end{frame}
\begin{frame}
	\frametitle{Exercise 1}

	\Large
	Find the limits: \\
	\vfill
	\begin{enumerate}\itemsep4ex
		\item $\displaystyle\lim_{x\to\infty}\left(\frac{3x^3-7}{2x^3-x+1}\right)$
		\item $\displaystyle\lim_{x\to\infty}\left(\frac{3x^2-7}{2x^3-x+1}\right)$
	\end{enumerate}
\end{frame}
\begin{frame}
	\frametitle{Exercise 1 - Solutions}

	\begin{enumerate}\itemsep2ex
		\item{
		            $\displaystyle\lim_{x\to\infty}\left(\frac{3x^3-7}{2x^3-x+1}\right)=\displaystyle\lim_{x\to\infty}\left(\frac{\frac{3\cancel{x^3}}{\cancel{x^3}}-\frac{7}{x^3}}{\frac{2\cancel{x^3}}{\cancel{x^3}}-\frac{x}{x^3}+\frac{1}{x^3}}\right)\newline=\displaystyle\lim_{x\to\infty}\left(\frac{3-\frac{7}{x^3}}{2-\frac{1}{x^2}+\frac{1}{x^3}}\right)=\frac{3-0}{2-0+0}=\boxed{\frac{3}{2}}$
		      }
		\item{
		            $\displaystyle\lim_{x\to\infty}\left(\frac{3x^2-7}{2x^3-x+1}\right)=\displaystyle\lim_{x\to\infty}\left(\frac{\frac{3x^2}{x^3}-\frac{7}{x^3}}{\frac{2\cancel{x^3}}{\cancel{x^3}}-\frac{x}{x^3}+\frac{1}{x^3}}\right)\newline=\displaystyle\lim_{x\to\infty}\left(\frac{\frac{3}{x}-\frac{7}{x^3}}{2-\frac{1}{x^2}+\frac{1}{x^3}}\right)=\frac{0-0}{2-0+0}=\boxed{0}$
		      }
	\end{enumerate}
\end{frame}
\begin{frame}
	\frametitle{Exercise 2}

	\Large
	Find the limit: \\
	\vfill
	\[\displaystyle\lim_{x\to-\infty}\left[\frac{1+e^x\sin(x)}{e^{x-1}-1}\right]\]
	\vfill
	\vfill
\end{frame}
\begin{frame}
	\frametitle{Exercise 2 - Solution}

	\Large
	\[\displaystyle\lim_{x\to-\infty}\left[\frac{1+e^x\sin(x)}{e^{x-1}-1}\right]=\frac{1+0}{0-1}\]
	\[=\frac{1}{-1}=\boxed{-1}\]
\end{frame}
\begin{frame}
	\frametitle{Exercise 3}

	\Large
	Find the limit: \\
	\vfill
	\[\displaystyle\lim_{x\to\infty}\left(\frac{3^x+2}{e^{2x}+1}\right)\]
	\vfill
	\vfill
\end{frame}
\begin{frame}
	\frametitle{Exercise 3 - Solution}

	\large
	\[\displaystyle\lim_{x\to\infty}\left(\frac{3^x+2}{e^{2x}+1}\right)=\displaystyle\lim_{x\to\infty}\left(\frac{\frac{3^x}{e^{2x}}+\frac{2}{e^{2x}}}{\frac{\cancel{e^{2x}}}{\cancel{e^{2x}}}+\frac{1}{e^{2x}}}\right)\]
	\[=\displaystyle\lim_{x\to\infty}\left[\frac{\left(\frac{3}{e^2}\right)^x+\frac{2}{e^{2x}}}{1+\frac{1}{e^{2x}}}\right]=\frac{0+0}{1+0}=\boxed{0}\]
\end{frame}
\end{document}