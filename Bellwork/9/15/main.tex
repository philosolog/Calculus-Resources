\documentclass[12pt]{beamer}
\usetheme{Warsaw}
\usepackage[utf8]{inputenc}
\usepackage{amsmath}
\usepackage{amsfonts}
\usepackage{amssymb}
\usepackage{graphicx}
\usepackage[font=Times,timeinterval=1,timeduration=2.0,timedeath=0,fillcolorwarningsecond=white!60!yellow,timewarningfirst=50,timewarningsecond=80,resetatpages=2]{tdclock}
\usepackage{tabularx}
\usepackage{array}
\usepackage{multicol}
\usepackage{longtable}
\usepackage{xcolor}
\usepackage{gensymb}
\usepackage{pgfplots}
\usepackage[makeroom]{cancel}

\graphicspath{ {./references/} }
\pgfplotsset{
	soldot/.style={color=black,only marks,mark=*},
	holdot/.style={color=black,fill=white,only marks,mark=*},
	compat=1.12
}
\newcolumntype{Y}{>{\centering\arraybackslash}X}
\makeatletter
\def\@listii{\leftmargin\leftmarginii
			  \topsep    2ex
			  \parsep    0\p@   \@plus\p@
			  \itemsep   \parsep}
\makeatother

\begin{document}
\begin{frame}
	\frametitle{Bellwork 9/15}
	\vspace*{\fill}
	\vspace*{\fill}
	\Large
	Evaluate:\\
	\large
	\initclock
	\begin{enumerate}
		\item \[\displaystyle\lim_{x\to4}\left(\frac{x^2-6x+8}{x-2}\right)\]
		\item \[\displaystyle\lim_{x\to9}\left(\frac{\sqrt{x}-3}{x-9}\right)\]
	\end{enumerate}
	\vspace*{\fill}
	\vspace*{\fill}
	\vspace*{\fill}
	\crono
	\resetcrono{\beamerbutton{reset}}
\end{frame}
\begin{frame}
	\frametitle{Bellwork 9/15 - Solutions}
	\begin{multicols}{2}
		\begin{enumerate}\itemsep6ex
			\large
			\item{
			            \[\displaystyle\lim_{x\to4}\left(\frac{x^2-6x+8}{x-2}\right)\]
			            \[=\displaystyle\lim_{x\to4}\left[\frac{(x-4)\cancel{(x-2)}}{\cancel{x-2}}\right]\]
			            \[=\displaystyle\lim_{x\to4}(x-4)\]
			            \[=\boxed{0}\]
			      }
			\small
			\item{
			            \[\displaystyle\lim_{x\to9}\left(\frac{\sqrt{x}-3}{x-9}\right)\]
			            \[=\displaystyle\lim_{x\to9}\left[\frac{\sqrt{x}-3}{(\sqrt{x})^2-3^2}\right]\]
			            \[=\displaystyle\lim_{x\to9}\left[\frac{\cancel{\sqrt{x}-3}}{\cancel{(\sqrt{x}-3)}(\sqrt{x}+3)}\right]\] % ?: Show canceling-out or leave it to be interpreted?
			            \[=\displaystyle\lim_{x\to9}\left(\frac{1}{\sqrt{x}+3}\right)\]
			            \[=\boxed{\frac{1}{6}}\]
			      }
		\end{enumerate}
	\end{multicols}
\end{frame}
\begin{frame}
	\frametitle{Exercise 1}
	Let $g$ and $h$ be the functions defined by: \[g(x) = \sin\left(\frac{\pi}{2}x\right)+4 \text{; } h(x) = -\frac{1}{4}x^3+\frac{3}{4}x+\frac{9}{2}\]\par
	If a function $f$ satisfies \[g(x)\leq f(x)\leq h(x) \text{ for } -1 < x < 2\text{,}\] what is $\displaystyle\lim_{x\to1}f(x)$?\par
	\vspace*{\fill}
	\begin{enumerate}\itemsep1ex
		\item $4$
		\item $\frac{9}{2}$
		\item $5$
		\item The limit cannot be determined from the information given.
	\end{enumerate}
\end{frame}
% \begin{frame}
% 	\frametitle{Exercise 1 - Solutions}

% \end{frame}
\begin{frame}
	\frametitle{Exercise 2}
	For all $x\neq0$, let $f$, $g$, and $h$ be the functions: \[f(x) = \frac{1-\cos(x)}{x^2} \text{, } g(x) = x^2\sin\left(\frac{1}{x}\right) \text{, and } h(x) = \frac{\sin(x)}{x}\]\par
	Which of the following inequalities can be used with the squeeze theorem to find the limit of the function as $x\to0$?\par
	\vspace*{\fill}
	\begin{enumerate}\itemsep2ex
		\item \hfill$\frac{1}{3}(1-x^2) \leq f(x)\leq \frac{1}{2}$\hfill~
		\item \hfill$-x^2\leq g(x)\leq x^2$\hfill~
		\item \hfill$-\frac{1}{|x|}\leq h(x)\leq \frac{1}{|x|}$\hfill~
	\end{enumerate}
	\vspace*{\fill}
	(All of the inequalities are true for $x\neq0$.)
\end{frame}
% \begin{frame}
% 	\frametitle{Exercise 2 - Solutions}

% \end{frame}
\begin{frame}
	\frametitle{Exercise 3}
	Let $g$ and $h$ be the functions defined by: \[g(x) = \sin\left[\frac{\pi}{2}(x+2)\right] + 3 \text{; } h(x) = \frac{1}{4}x^3-\frac{3}{2}x^2-\frac{9}{4}x+3\]\par
	If $f$ is a function that satisfies \[g(x)\leq f(x)\leq h(x) \text{ for } -2 < x < 0\text{,}\]\par
	what is $\displaystyle\lim_{x\to-1}f(x)$?
	\begin{enumerate}\itemsep1ex
		\item $3$
		\item $3.5$
		\item $4$
		\item The limit cannot be determined from the information given.
	\end{enumerate}
\end{frame}
% \begin{frame}
% 	\frametitle{Exercise 3 - Solutions}

% \end{frame}
\end{document}