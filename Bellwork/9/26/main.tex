\documentclass[12pt]{beamer}
\usetheme{Warsaw}
\usepackage[utf8]{inputenc}
\usepackage{amsmath}
\usepackage{amsfonts}
\usepackage{amssymb}
\usepackage{graphicx}
\usepackage[font=Times,timeinterval=1,timeduration=2.0,timedeath=0,fillcolorwarningsecond=white!60!yellow,timewarningfirst=50,timewarningsecond=80,resetatpages=2]{tdclock}
\usepackage{tabularx}
\usepackage{array}
\usepackage{multicol}
\usepackage{longtable}
\usepackage{xcolor}
\usepackage{gensymb}
\usepackage{pgfplots}
\usepackage[makeroom]{cancel}

\graphicspath{ {./references/} }
\pgfplotsset{
	soldot/.style={color=black,only marks,mark=*},
	holdot/.style={color=black,fill=white,only marks,mark=*},
	compat=1.12
}
\newcolumntype{Y}{>{\centering\arraybackslash}X}
\makeatletter
\def\@listii{\leftmargin\leftmarginii
			  \topsep    2ex
			  \parsep    0\p@   \@plus\p@
			  \itemsep   \parsep}
\makeatother

\begin{document}
\begin{frame}
	\frametitle{Bellwork 9/26}
	\initclock

	\vfill
	\vfill
	\Large
	Find the limits:\par
	\[\displaystyle\lim_{x\to \infty}\frac{\sqrt{1+4x^6}}{2-x^3}\text{ and }\displaystyle\lim_{x\to -\infty}\frac{\sqrt{1+4x^6}}{2-x^3}\]
	\vfill
	\vfill
	\vfill

	\small
	\crono
	\resetcrono{\beamerbutton{reset}}
\end{frame}
\begin{frame}
	\frametitle{Bellwork 9/26 - Solution, Part 1}

	For $x\to \infty$, we use $x^3 = +\sqrt{x^6}$:
	\[\displaystyle\lim_{x\to \infty}\left(\frac{\sqrt{1+4x^6}}{2-\sqrt{x^6}}\right)=\displaystyle\lim_{x\to \infty}\left(\frac{\frac{\sqrt{1+4x^6}}{x^3}}{\frac{2}{x^3}-1}\right)=\displaystyle\lim_{x\to \infty}\left(\frac{\sqrt{\frac{1}{x^6}+\frac{4\cancel{x^6}}{\cancel{x^6}}}}{\frac{2}{x^3}-1}\right)\]
	\[=\boxed{-2}\]
\end{frame}
\begin{frame}
	\frametitle{Bellwork 9/26 - Solution, Part 2}

	For $x\to -\infty$, we use $x^3 = -\sqrt{x^6}$:
	\[\displaystyle\lim_{x\to \infty}\left(\frac{\sqrt{1+4x^6}}{2+\sqrt{x^6}}\right)=\displaystyle\lim_{x\to \infty}\left(\frac{\frac{\sqrt{1+4x^6}}{x^3}}{\frac{2}{x^3}+1}\right)=\displaystyle\lim_{x\to \infty}\left(\frac{\sqrt{\frac{1}{x^6}+\frac{4\cancel{x^6}}{\cancel{x^6}}}}{\frac{2}{x^3}+1}\right)\]
	\[=\boxed{2}\]
\end{frame}
\begin{frame}
	\frametitle{Exercise 1}

	\vfill
	\LARGE
	\[f(x) = \frac{1}{x+1}\]
	\vfill
	\Large
	\[f'(2)=\text{?}\]
	\vfill
	\vfill
\end{frame}
\begin{frame}
	\frametitle{Exercise 1 - Solution}

	\large % ?: "Recall" and then the rule?
	\[f'(2)=\displaystyle\lim_{x\to2}\left[\frac{f(x)-f(2)}{x-2}\right]\]
	\[\implies f'(2)=\displaystyle\lim_{x\to2}\left(\frac{\frac{1}{x+1}-\frac{1}{2+1}}{x-2}\right)=\displaystyle\lim_{x\to2}\left[\frac{\frac{3-(x+1)}{3(x+1)}}{x-2}\right]\]
	\[=\displaystyle\lim_{x\to2}\left[\frac{-\cancel{(x-2)}}{3(x+1)\cancel{(x-2)}}\right]=\displaystyle\lim_{x\to2}\left[\frac{-1}{3(x+1)}\right]=\boxed{-\frac{1}{9}}\]
\end{frame}
\begin{frame}
	\frametitle{Exercise 2}

	\vfill
	\LARGE
	\[f(x) = \sqrt{2x+1}\]
	\vfill
	\Large
	\[f'(4)=\text{?}\]
	\vfill
	\vfill
\end{frame}
\begin{frame}
	\frametitle{Exercise 2 - Solution}

	\large
	\[f'(4)=\displaystyle\lim_{x\to4}\left[\frac{f(x)-f(4)}{x-4}\right]\]
	\vfill
	\small
	\[\implies f'(4)=\displaystyle\lim_{x\to4}\left(\frac{\sqrt{2x+1}-\sqrt{2\cdot4+1}}{x-4}\right)\]
	\[=\displaystyle\lim_{x\to4}\left(\frac{\sqrt{2x+1}-3}{x-4}\right)=\displaystyle\lim_{x\to4}\left[\frac{(\sqrt{2x+1}-3)}{(x-4)}\cdot \frac{(\sqrt{2x+1}+3)}{(\sqrt{2x+1}+3)}\right]\]
	\tiny
	\[=\displaystyle\lim_{x\to4}\left[\frac{2x+1-9}{(x-4)\sqrt{2x+1}+3(x-4)}\right]=\displaystyle\lim_{x\to4}\left[\frac{2\cancel{(x-4)}}{\cancel{(x-4)}\sqrt{2x+1}+3\cancel{(x-4)}}\right]\]
	\small
	\[=\displaystyle\lim_{x\to4}\left[\frac{2}{\sqrt{2x+1}+3}\right]=\displaystyle\lim_{x\to4}\left[\frac{2}{\sqrt{2x+1}+3}\right]=\boxed{\frac{1}{3}}\]
	\vfill
\end{frame}
\end{document}