\documentclass[12pt]{beamer}
\usetheme{Warsaw}
\usepackage[utf8]{inputenc}
\usepackage{amsmath}
\usepackage{amsfonts}
\usepackage{amssymb}
\usepackage{graphicx}
\usepackage[font=Times,timeinterval=1,timeduration=2.0,timedeath=0,fillcolorwarningsecond=white!60!yellow,timewarningfirst=50,timewarningsecond=80,resetatpages=2]{tdclock}
\usepackage{tabularx}
\usepackage{array}
\usepackage{multicol}
\usepackage{longtable}
\usepackage{xcolor}
\usepackage{gensymb}
\usepackage{pgfplots}
\usepackage[makeroom]{cancel}

\graphicspath{ {./references/} }
\pgfplotsset{
	soldot/.style={color=black,only marks,mark=*},
	holdot/.style={color=black,fill=white,only marks,mark=*},
	compat=1.12
}
\newcolumntype{Y}{>{\centering\arraybackslash}X}
\makeatletter
\def\@listii{\leftmargin\leftmarginii
			  \topsep    2ex
			  \parsep    0\p@   \@plus\p@
			  \itemsep   \parsep}
\makeatother

\begin{document}
\begin{frame}
	\frametitle{Bellwork 9/19}
	\initclock
	\vfill
	\vfill
	\vfill
	\Large
	Find the values $a$ and $b$ that make $f(x)$ continuous.
	\[
		f(x)=\begin{cases}
			\begin{aligned}[t]
				ax^2 + b     & \text{ if } x \leq 0     \\
				bx^2 - x + a & \text{ if } 0 < x \leq 2 \\
				x + 3b       & \text{ if } x > 2
			\end{aligned}
		\end{cases}
	\]\par
	\vfill
	\vfill
	\vfill
	\vfill
	\small
	\crono
	\resetcrono{\beamerbutton{reset}}
\end{frame}
\begin{frame}
	\frametitle{Bellwork 9/19 - Solution}
	% https://www.desmos.com/calculator/xiebmtic9w

	\large
	\begin{minipage}{0.5\textwidth}
		\begin{center}
			$\begin{aligned}[t]
				a(0) + b & = b(0) - 0 + a \\
				b & = a
			\end{aligned}$
		\end{center}
	\end{minipage}%
	\begin{minipage}{0.5\textwidth}
		\begin{center}
			$\begin{aligned}[t]
				4b - 2 + a & = 2 + 3b\\
				4 = b + a
			\end{aligned}$
		\end{center}
	\end{minipage}\\
	\[
		\implies\begin{aligned}[t]
			4 & = 2b \\
			~\\
			b & = 2 \\
			a & = 2
		\end{aligned}
	\]
\end{frame}
\begin{frame}
	\frametitle{Exercise 1}
	\vfill
	\vfill
	\vfill
	\vfill
	Sketch a function with all of the following properties:\par
	\vfill
	\begin{enumerate}\itemsep2ex
		\item Continuous everywhere except at $x = -1$ and $x = 4$.
		\item Vertical asymptote at $x = -1$.
		\item Removable discontinuity at $x = 4$.
	\end{enumerate}
	\vfill
	\vfill
	\vfill
	\vfill
	\vfill
\end{frame}
\begin{frame}
	\frametitle{Exercise 2}
	\vfill
	\vfill
	\vfill
	\vfill
	Sketch a function with all of the following properties:\par
	\vfill
	\begin{enumerate}\itemsep2ex
		\item Continuous on $x > -1$.
		\item Jump discontinuity at $x = -3$.
		\item Vertical asymptote at $x = -1$.
		\item Always positive when $x < 0$.
	\end{enumerate}
	\vfill
	\vfill
	\vfill
	\vfill
	\vfill
\end{frame}
\begin{frame}
	\frametitle{Exercise 3}
	\vfill
	\vfill
	\vfill
	\vfill
	Sketch a function with all of the following properties:\par
	\vfill
	\begin{enumerate}\itemsep2ex
		\item Neither left nor right continuous at $x = -1$.
		\item Vertical asymptote at $x = 1$.
		\item Oscillating discontinuity at $x = 2$.
		\item Continuous everywhere except at $x = -1\text{, } 1\text{, } 2$.
	\end{enumerate}
	\vfill
	\vfill
	\vfill
	\vfill
	\vfill
\end{frame}
\begin{frame}
	\frametitle{Exercise 4}
	\Large
	Where is $h(x) = \ln[1+\sin(x)]$ continuous?
\end{frame}
\begin{frame}
	\frametitle{Exercise 4 - Solution}
	% https://www.desmos.com/calculator/d6cs67vpmn

	\large
	\[h(x) = \ln[1+\sin(x)]\]\par
	\vfill
	\small
	In this case, $h(x)$ is continuous if the output of $\ln(...)$ is defined.
	\[
		\implies\begin{aligned}[t]
			1+\sin(x) & > 0  \\
			\sin(x)   & > -1 \\
		\end{aligned}
	\]\par
	Since the range of $\sin(x)$ is $[-1, 1]$, we only need to find $x$ where $\sin(x)\neq-1$ in order to find where $h(x)$ is undefined.
	\[
		\implies\begin{aligned}[t]
			x \neq 2\pi n - \frac{\pi}{2} \text{ and } x \neq 2\pi n - \frac{3\pi}{2} \text{ where } n\in \mathbb{Z} \\
		\end{aligned}
	\]\par
	Therefore, $h(x)$ is continuous everywhere besides \[x = 2\pi n - \frac{\pi}{2} \text{, } 2\pi n - \frac{3\pi}{2} \text{ where } n\in \mathbb{Z}\]
\end{frame}
\begin{frame}
	\frametitle{Exercise 5}
	\Large
	Evaluate: \[\displaystyle\lim_{x\to1}\left[\arcsin\left(\frac{1-\sqrt{x}}{1-x}\right)\right]\]\par
\end{frame}
\begin{frame}
	\frametitle{Exercise 5 - Solution}
	Because $\arcsin$ is a continuous function,\par
	\[\displaystyle\lim_{x\to1}\left[\arcsin\left(\frac{1-\sqrt{x}}{1-x}\right)\right]=\arcsin\left[\displaystyle\lim_{x\to1}\left(\frac{1-\sqrt{x}}{1-x}\right)\right]\]\par
	\[\arcsin\left[\displaystyle\lim_{x\to1}\left(\frac{1-\sqrt{x}}{1-x}\right)\right]=\arcsin\left\{\displaystyle\lim_{x\to1}\left[\frac{\cancel{1-\sqrt{x}}}{\cancel{(1-\sqrt{x})}(1+\sqrt{x})}\right]\right\}\]\par
	\[=\arcsin\left[\displaystyle\lim_{x\to1}\left(\frac{1}{1+\sqrt{x}}\right)\right]=\arcsin\left(\frac{1}{2}\right)=\boxed{\frac{\pi}{6}}\]
\end{frame}
\end{document}