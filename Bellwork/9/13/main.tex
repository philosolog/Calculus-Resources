\documentclass[12pt]{beamer}
\usetheme{Warsaw}
\usepackage[utf8]{inputenc}
\usepackage{amsmath}
\usepackage{amsfonts}
\usepackage{amssymb}
\usepackage{graphicx}
\usepackage[font=Times,timeinterval=1,timeduration=2.0,timedeath=0,fillcolorwarningsecond=white!60!yellow,timewarningfirst=50,timewarningsecond=80,resetatpages=2]{tdclock}
\usepackage{tabularx}
\usepackage{array}
\usepackage{multicol}
\usepackage{longtable}
\usepackage{xcolor}
\usepackage{gensymb}
\usepackage{pgfplots}

\graphicspath{ {./references/} }
\pgfplotsset{
	soldot/.style={color=black,only marks,mark=*},
	holdot/.style={color=black,fill=white,only marks,mark=*},
	compat=1.12
}
\newcolumntype{Y}{>{\centering\arraybackslash}X} % *: Beamer enumeration spacing-workaround..
\makeatletter
\def\@listii{\leftmargin\leftmarginii
			  \topsep    2ex
			  \parsep    0\p@   \@plus\p@
			  \itemsep   \parsep}
\makeatother

\begin{document}
\begin{frame}
	\frametitle{Bellwork 9/13}
	\vspace*{\fill}
	\vspace*{\fill}
	\initclock
	\begin{enumerate}
		\item Evaluate \[\displaystyle\lim_{x\to5}\left(\frac{x^2+x-30}{5-x}\right)\]
		\item Find $\displaystyle\lim_{x\to0}f(x)$ where
		      \[
			      f(x) =
			      \begin{cases}
				      \sqrt{4-x} & \text{if } x < 0    \\
				      x+2        & \text{if } x \geq 0 \\
			      \end{cases}
		      \]\\
	\end{enumerate}
	\vspace*{\fill}
	\vspace*{\fill}
	\crono
	\resetcrono{\beamerbutton{reset}}
\end{frame}
\begin{frame}
	\frametitle{Bellwork 9/13 - Solutions}
	\begin{enumerate}\itemsep2ex
		\item $\displaystyle\lim_{x\to5}\left(\frac{x^2+x-30}{5-x}\right)=\boxed{-11}$
		\item
		      \[
			      f(x) =
			      \begin{cases}
				      \sqrt{4-x} & \text{if } x < 0    \\
				      x+2        & \text{if } x \geq 0 \\
			      \end{cases}
		      \]\\
		      \begin{table}[]
			      $\displaystyle\lim_{x\to0^{-}}f(x)=2$
			      \hspace{0.25cm}
			      $\displaystyle\lim_{x\to0^{+}}f(x)=2$
		      \end{table}
		      \vspace*{\fill}
		      \[\implies\displaystyle\lim_{x\to0}f(x) = \boxed{2}\]
	\end{enumerate}
\end{frame}
\begin{frame}
	\frametitle{Exercise 1}
	\vspace*{\fill}
	\vspace*{\fill}
	\vspace*{\fill}
	\initclock
	\Large
	\[\displaystyle\lim_{x\to2}\left(\frac{x+3}{x^2+x-6}\right)\]\\
	\vspace*{\fill}
	\vspace*{\fill}
	\vspace*{\fill}
	\vspace*{\fill}
	\crono
	\resetcrono{\beamerbutton{reset}}
\end{frame}
\end{document}