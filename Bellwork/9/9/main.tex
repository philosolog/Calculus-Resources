\documentclass[12pt]{beamer}
\usetheme{Warsaw}
\usepackage[utf8]{inputenc}
\usepackage{amsmath}
\usepackage{amsfonts}
\usepackage{amssymb}
\usepackage{graphicx}
\usepackage[font=Times,timeinterval=1,timeduration=2.0,timedeath=0,fillcolorwarningsecond=white!60!yellow,timewarningfirst=50,timewarningsecond=80,resetatpages=2]{tdclock}
\usepackage{tabularx}
\usepackage{array}
\usepackage{multicol}
\usepackage{longtable}
\usepackage{xcolor}
\usepackage{gensymb}
\usepackage{pgfplots}

\pgfplotsset{
	soldot/.style={color=black,only marks,mark=*},
	holdot/.style={color=black,fill=white,only marks,mark=*},
	compat=1.12
}
\newcolumntype{Y}{>{\centering\arraybackslash}X}
\begin{document}
\begin{frame}
	\frametitle{Bellwork 9/9}
	\vspace*{\fill}
	\vspace*{\fill}
	\initclock
	If a rock is thrown upward on the planet Mars with a velocity of 10 $\tfrac{m}{s}$, its height in meters $t$ seconds later is given by: \[y=10t-1.86t^2\]
	\vspace*{\fill}
	\begin{enumerate}
		\item Find the average velocity over the given time intervals:
		      \begin{multicols}{3}
			      \begin{enumerate}
				      \item $[1, 1.1]$
				      \item $[1, 1.01]$
				      \item $[1, 1.001]$
			      \end{enumerate}
		      \end{multicols}
		      \vspace*{\fill}
		\item Estimate the instantaneous velocity when $t=1$.
	\end{enumerate}
	\vspace*{\fill}
	\vspace*{\fill}
	\vspace*{\fill}
	\vspace*{\fill}
	\crono
	\resetcrono{\beamerbutton{reset}}
\end{frame}
\begin{frame}
	\frametitle{Bellwork 9/9 - Solutions}
	\vspace*{\fill}
	\vspace*{\fill}
	\begin{enumerate}
		\item
		      \begin{multicols}{3}
			      \begin{enumerate}
				      \item $6.094\tfrac{m}{s}$
				      \item $6.2614\tfrac{m}{s}$
				      \item $6.27814\tfrac{m}{s}$
			      \end{enumerate}
		      \end{multicols}
		      \vspace*{\fill}
		\item $6.28\tfrac{m}{s}$
	\end{enumerate}
	\vspace*{\fill}
	\vspace*{\fill}
	\vspace*{\fill}
	\vspace*{\fill}
\end{frame}
\begin{frame}
	\frametitle{Exercise 1}
	\initclock
	\large
	\begin{enumerate}
		\item Graph $f(x) = \frac{1-\cos(x)}{x}$ on a calculator.
		      \vspace*{\fill}
		      \vspace*{\fill}
		\item Fill in the table:
		      \small
		      \begin{table}[]
			      \begin{tabular}{c|c}
				      $x$     & $f(x)$ \\ \hline
				      0.5   &        \\
				      0.1   &        \\
				      0.01  &        \\
				      0.001 &
			      \end{tabular}
		      \end{table}
		      \large
		      \vspace*{\fill}
		      \vspace*{\fill}
		\item Estimate $\displaystyle\lim_{x\to 0}\left[\frac{1-\cos(x)}{x}\right]$
	\end{enumerate}
	\vspace*{\fill}
	\vspace*{\fill}
	\vspace*{\fill}
	\vspace*{\fill}
	\crono
	\resetcrono{\beamerbutton{reset}}
\end{frame}
\begin{frame}
	\frametitle{Exercise 1 - Solutions}
	\vspace*{\fill}
	\vspace*{\fill}
	\begin{enumerate}
		\item \begin{center}
			      \begin{tikzpicture}
				      \begin{axis}
					      [
						      xlabel={x},
						      ylabel={y},
						      scale only axis,
						      xmin=-2,
						      xmax=2,
						      ymin=-2,
						      ymax=2,
						      xtick={-2,-1,...,2},
						      ytick={-2,-1,...,2},
						      axis lines=center
					      ]
					      \addplot[no marks, domain=-4:4, samples=100] {(1-cos(\x r))/x};
					      \addplot[holdot] coordinates{(0,0)};
				      \end{axis}
			      \end{tikzpicture}

		      \end{center}
	\end{enumerate}
	\vspace*{\fill}
	\vspace*{\fill}
	\vspace*{\fill}
	\vspace*{\fill}
\end{frame}
\begin{frame}
	\frametitle{Exercise 1 - Solutions} % ?: Should I specify ", Continued" here?
	\begin{multicols}{2}
		\begin{enumerate}
			\setcounter{enumi}{1}
			\large

			\item \begin{table}[]
				      \begin{tabular}{c|c}
					      $x$     & $f(x)$ \\ \hline
					      -0.1  & -0.05  \\
					      -0.01 & -0.005 \\
					      0.01  & 0.005  \\
					      0.1   & 0.05
				      \end{tabular}
			      \end{table}
			\item From the table, \[\displaystyle\lim_{x\to 0}\left[\frac{1-\cos(x)}{x}\right]\approx \boxed{0}\]
		\end{enumerate}
	\end{multicols}
\end{frame}
\begin{frame}
	\frametitle{Exercise 2}
	\initclock
	\large
	\begin{enumerate}
		\item Graph $f(x) = \frac{x^2-9}{x+3}$ on a calculator.
		      \vspace*{\fill}
		      \vspace*{\fill}
		\item Fill in the table:
		      \small
		      \begin{table}[]
			      \begin{tabular}{c|c}
				      $x$     & $f(x)$ \\ \hline
				      -3.1  &        \\
				      -3.01 &        \\
				      -2.99 &        \\
				      -2.9  &
			      \end{tabular}
		      \end{table}
		      \large
		      \vspace*{\fill}
		      \vspace*{\fill}
		\item Estimate $\displaystyle\lim_{x\to -3}\left(\frac{x^2-9}{x+3}\right)$
	\end{enumerate}
	\vspace*{\fill}
	\vspace*{\fill}
	\vspace*{\fill}
	\vspace*{\fill}
	\crono
	\resetcrono{\beamerbutton{reset}}
\end{frame}
\begin{frame}
	\frametitle{Exercise 2 - Solutions}
	\vspace*{\fill}
	\vspace*{\fill}
	\begin{enumerate}
		\item \begin{center}
			      \begin{tikzpicture}
				      \begin{axis}
					      [
						      xlabel={x},
						      ylabel={y},
						      axis lines=center
					      ]
					      \addplot[no marks, domain=-4:4, samples=100] {(x^2-9)/(x+3)};
					      \addplot[holdot] coordinates{(-3,-6)};
				      \end{axis}
			      \end{tikzpicture}

		      \end{center}
	\end{enumerate}
	\vspace*{\fill}
	\vspace*{\fill}
	\vspace*{\fill}
	\vspace*{\fill}
\end{frame}
\begin{frame}
	\frametitle{Exercise 2 - Solutions} % ?: Should I specify ", Continued" here as well?
	\begin{multicols}{2}
		\begin{enumerate}
			\setcounter{enumi}{1}
			\large

			\item \begin{table}[]
				      \begin{tabular}{c|c}
					      $x$       & $f(x)$ \\ \hline
					      -3.1    & -6.1  \\
					      -3.01   & -6.01 \\
					      -2.99   & -5.99  \\
					      -2.90.1 & -5.9
				      \end{tabular}
			      \end{table}
			\item From the table, \[\displaystyle\lim_{x\to -3}\left(\frac{x^2-9}{x+3}\right)\approx \boxed{-6}\]
		\end{enumerate}
	\end{multicols}
\end{frame}
\end{document}