\documentclass[12pt]{beamer}
\usetheme{Warsaw}
\usepackage[utf8]{inputenc}
\usepackage{amsmath}
\usepackage{amsfonts}
\usepackage{amssymb}
\usepackage{graphicx}
\usepackage[font=Times,timeinterval=1,timeduration=2.0,timedeath=0,fillcolorwarningsecond=white!60!yellow,timewarningfirst=50,timewarningsecond=80,resetatpages=2]{tdclock}
\usepackage{tabularx}
\usepackage{array}
\usepackage{multicol}
\usepackage{longtable}
\usepackage{xcolor}
\usepackage{gensymb}
\usepackage{pgfplots}

\graphicspath{ {./references/} }
\pgfplotsset{
	soldot/.style={color=black,only marks,mark=*},
	holdot/.style={color=black,fill=white,only marks,mark=*},
	compat=1.12
}
\newcolumntype{Y}{>{\centering\arraybackslash}X} % *: Beamer enumeration spacing-workaround..
\makeatletter
\def\@listii{\leftmargin\leftmarginii
			  \topsep    2ex
			  \parsep    0\p@   \@plus\p@
			  \itemsep   \parsep}
\makeatother

\begin{document}
\begin{frame}
	\frametitle{Bellwork 9/12}
	\vspace*{\fill}
	\vspace*{\fill}
	\initclock
	\large
	Evaluate without a calculator:\\
	\vspace*{\fill}
	\vspace*{\fill}
	\begin{center}
		\begin{minipage}{0.8\textwidth}
			\begin{enumerate}\itemsep4ex % *: See line 23.
				\item $\displaystyle\lim_{x\to3^{-}}\left(\frac{x}{x+3}\right) \text{ and } \displaystyle\lim_{x\to3^{+}}\left(\frac{x}{x+3}\right)$
				\item $\displaystyle\lim_{x\to\pi^{-}}\left[\frac{\cos(x)}{\sin(x)}\right] \text{ and } \displaystyle\lim_{x\to\pi^{+}}\left[\frac{\cos(x)}{\sin(x)}\right]$
			\end{enumerate}
		\end{minipage}
	\end{center}
	\vspace*{\fill}
	\vspace*{\fill}
	\vspace*{\fill}
	\crono
	\resetcrono{\beamerbutton{reset}}
\end{frame}
\begin{frame}
	\frametitle{Bellwork 9/12 - Solutions}
	\begin{center}
		\begin{minipage}{1\linewidth}
			\begin{enumerate}
				\item ~
				      \begin{enumerate}\itemsep2ex
					      \item $\displaystyle\lim_{x\to3^{-}}\left(\frac{x}{x+3}\right) = \frac{1}{2}$
					      \item $\displaystyle\lim_{x\to3^{+}}\left(\frac{x}{x+3}\right) = \frac{1}{2}$
				      \end{enumerate}
				\item ~
				      \begin{enumerate}\itemsep2ex
					      \item $\displaystyle\lim_{x\to\pi^{-}}\left[\frac{\cos(x)}{\sin(x)}\right] = -\infty$
					      \item $\displaystyle\lim_{x\to\pi^{+}}\left[\frac{\cos(x)}{\sin(x)}\right] = \infty$
				      \end{enumerate}
			\end{enumerate}
		\end{minipage}
	\end{center}
\end{frame}
\begin{frame}
	\frametitle{Exercise 1}
	\vspace*{\fill}
	\vspace*{\fill}
	\vspace*{\fill}
	\initclock
	\Large
	\[\displaystyle\lim_{x\to2}\left(\frac{x+3}{x^2+x-6}\right)\]\\
	\vspace*{\fill}
	\vspace*{\fill}
	\vspace*{\fill}
	\vspace*{\fill}
	\crono
	\resetcrono{\beamerbutton{reset}}
\end{frame}
\begin{frame}
	\frametitle{Exercise 1 - Solution}
	\vspace*{\fill}
	\Large
	\[\displaystyle\lim_{x\to2}\left(\frac{x+3}{x^2+x-6}\right) \text{ }\boxed{\text{DNE}}\]\\
	\small
	\vspace*{\fill}
	\vspace*{\fill}
	\begin{minipage}{0.5\textwidth}
		$\displaystyle\lim_{x\to2^{-}}\left(\frac{x+3}{x^2+x-6}\right)=-\infty$
	\end{minipage}%
	\begin{minipage}{0.5\textwidth}
		$\displaystyle\lim_{x\to2^{+}}\left(\frac{x+3}{x^2+x-6}\right)=\infty$
	\end{minipage}\\
	\vspace*{\fill}
	\vspace*{\fill}
\end{frame}
\begin{frame}
	\frametitle{Exercise 2}
	\vspace*{\fill}
	\vspace*{\fill}
	\vspace*{\fill}
	\initclock
	\Large
	\[\displaystyle\lim_{x\to1}\left(\frac{x-1}{x^2+3x+2}\right)\]\\
	\vspace*{\fill}
	\vspace*{\fill}
	\vspace*{\fill}
	\vspace*{\fill}
	\crono
	\resetcrono{\beamerbutton{reset}}
\end{frame}
\begin{frame}
	\frametitle{Exercise 2 - Solution}
	\Large
	\[\displaystyle\lim_{x\to1}\left(\frac{x-1}{x^2+3x+2}\right) = \boxed{0}\]
\end{frame}
\begin{frame}
	\frametitle{Exercise 3}
	\vspace*{\fill}
	\vspace*{\fill}
	\vspace*{\fill}
	\initclock
	\Large
	\[\displaystyle\lim_{h\to0}\left(\frac{\sqrt{4+h}-2}{h}\right)\]\\
	\vspace*{\fill}
	\vspace*{\fill}
	\vspace*{\fill}
	\vspace*{\fill}
	\crono
	\resetcrono{\beamerbutton{reset}}
\end{frame}
\begin{frame}
	\frametitle{Exercise 3 - Solution}
	\Large
	\[\displaystyle\lim_{h\to0}\left(\frac{\sqrt{4+h}-2}{h}\right) = \boxed{\frac{1}{4}}\]
\end{frame}
\begin{frame}
	\frametitle{Exercise 4}
	\vspace*{\fill}
	\vspace*{\fill}
	\vspace*{\fill}
	\initclock
	\Large
	\[\displaystyle\lim_{h\to0}\left[\frac{(4+h)^2-16}{h}\right]\]\\
	\vspace*{\fill}
	\vspace*{\fill}
	\vspace*{\fill}
	\vspace*{\fill}
	\crono
	\resetcrono{\beamerbutton{reset}}
\end{frame}
\begin{frame}
	\frametitle{Exercise 4 - Solution}
	\Large
	\[\displaystyle\lim_{h\to0}\left[\frac{(4+h)^2-16}{h}\right] = \boxed{8}\]
\end{frame}
\begin{frame}
	\frametitle{Exercise 5}
	\initclock
	\Large
	\begin{center}
		$f(x) =
			\begin{cases}
				x+3   & \text{if } x < 3    \\
				x^2-2 & \text{if } x \geq 3 \\
			\end{cases}$\\
		\vspace*{\fill}
		\vspace*{\fill}
		\vspace*{\fill}
		Find $\displaystyle\lim_{x\to3}f(x)$\\
	\end{center}
	\vspace*{\fill}
	\vspace*{\fill}
	\crono
	\resetcrono{\beamerbutton{reset}}
\end{frame}
\begin{frame}
	\frametitle{Exercise 5 - Solution}
	\large
	\begin{center}
		$f(x) =
			\begin{cases}
				x+3   & \text{if } x < 3    \\
				x^2-2 & \text{if } x \geq 3 \\
			\end{cases}$\\
		\vspace*{\fill}
		\vspace*{\fill}
		\begin{table}[]
			$\displaystyle\lim_{x\to3^{-}}f(x)=6$
			\hspace{0.25cm}
			$\displaystyle\lim_{x\to3^{+}}f(x)=7$
		\end{table}
		\vspace*{\fill}
		\[\implies\displaystyle\lim_{x\to3}f(x) \text{ }\boxed{\text{DNE}}\]
	\end{center}
	\vspace*{\fill}
	\vspace*{\fill}
\end{frame}
\begin{frame}
	\frametitle{Exercise 6}
	\initclock
	\Large
	\begin{center}
		$g(x) =
			\begin{cases}
				\sqrt{x-2} & \text{if } x \leq 3 \\
				2x-5       & \text{if } x > 3    \\
			\end{cases}
		$\\
		\vspace*{\fill}
		\vspace*{\fill}
		\vspace*{\fill}
		Find $\displaystyle\lim_{x\to3}g(x)$\\
	\end{center}
	\vspace*{\fill}
	\vspace*{\fill}
	\crono
	\resetcrono{\beamerbutton{reset}}
\end{frame}
\begin{frame}
	\frametitle{Exercise 6 - Solution}
	\large
	\begin{center}
		$g(x) =
			\begin{cases}
				\sqrt{x-2} & \text{if } x \leq 3 \\
				2x-5       & \text{if } x > 3    \\
			\end{cases}
		$\\
		\vspace*{\fill}
		\vspace*{\fill}
		\begin{table}[]
			$\displaystyle\lim_{x\to3^{-}}g(x)=1$
			\hspace{0.25cm}
			$\displaystyle\lim_{x\to3^{+}}g(x)=1$
		\end{table}
		\vspace*{\fill}
		\[\implies\displaystyle\lim_{x\to3}g(x) = \boxed{1}\]
	\end{center}
	\vspace*{\fill}
	\vspace*{\fill}
\end{frame}
\end{document}