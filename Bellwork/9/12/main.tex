\documentclass[12pt]{beamer}
\usetheme{Warsaw}
\usepackage[utf8]{inputenc}
\usepackage{amsmath}
\usepackage{amsfonts}
\usepackage{amssymb}
\usepackage{graphicx}
\usepackage[font=Times,timeinterval=1,timeduration=2.0,timedeath=0,fillcolorwarningsecond=white!60!yellow,timewarningfirst=50,timewarningsecond=80,resetatpages=2]{tdclock}
\usepackage{tabularx}
\usepackage{array}
\usepackage{multicol}
\usepackage{longtable}
\usepackage{xcolor}
\usepackage{gensymb}
\usepackage{pgfplots}

\graphicspath{ {./references/} }
\pgfplotsset{
	soldot/.style={color=black,only marks,mark=*},
	holdot/.style={color=black,fill=white,only marks,mark=*},
	compat=1.12
}
\newcolumntype{Y}{>{\centering\arraybackslash}X}
\begin{document}
\begin{frame}
	\frametitle{Bellwork 9/12}
	\vspace*{\fill}
	\vspace*{\fill}
	\initclock
	Evaluate without a calculator:\\
	\vspace*{\fill}
	\begin{center}
		\large
		\begin{minipage}{0.75\textwidth}
			\begin{enumerate}
				\item $\displaystyle\lim_{x\to3^{-}}\left(\frac{x}{x+3}\right) \text{ and } \displaystyle\lim_{x\to3^{+}}\left(\frac{x}{x+3}\right)$
				% TODO: Create extra space.
				\item $\displaystyle\lim_{x\to\pi^{-}}\left[\frac{\cos(x)}{\sin(x)}\right] \text{ and } \displaystyle\lim_{x\to\pi^{+}}\left[\frac{\cos(x)}{\sin(x)}\right]$
			\end{enumerate}
		\end{minipage}
	\end{center}
	\vspace*{\fill}
	\vspace*{\fill}
	\vspace*{\fill}
	\vspace*{\fill}
	\crono
	\resetcrono{\beamerbutton{reset}}
\end{frame}
\begin{frame}
	\frametitle{Bellwork 9/12 - Solutions}

\end{frame}
\begin{frame}
	\frametitle{Exercise 1}
	\vspace*{\fill}
	\vspace*{\fill}
	\vspace*{\fill}
	\initclock
	\Large
	\[\displaystyle\lim_{x\to2}\left(\frac{x+3}{x^2+x-6}\right)\]\\
	\vspace*{\fill}
	\vspace*{\fill}
	\vspace*{\fill}
	\vspace*{\fill}
	\crono
	\resetcrono{\beamerbutton{reset}}
\end{frame}
\begin{frame}
	\frametitle{Exercise 1 - Solution}
	\vspace*{\fill}
	\initclock
	\Large
	\[\displaystyle\lim_{x\to2}\left(\frac{x+3}{x^2+x-6}\right) \text{ }\boxed{\text{DNE}}\]\\
	\small
	\vspace*{\fill}
	\vspace*{\fill}
	\begin{minipage}{0.5\textwidth}
		$\displaystyle\lim_{x\to2^{-}}\left(\frac{x+3}{x^2+x-6}\right)=-\infty$
	\end{minipage}%
	\begin{minipage}{0.5\textwidth}
		$\displaystyle\lim_{x\to2^{+}}\left(\frac{x+3}{x^2+x-6}\right)=\infty$
	\end{minipage}\\
	\vspace*{\fill}
	\vspace*{\fill}
	\crono
	\resetcrono{\beamerbutton{reset}}
\end{frame}
\begin{frame}
	\frametitle{Exercise 2}
	\vspace*{\fill}
	\vspace*{\fill}
	\vspace*{\fill}
	\initclock
	\Large
	\[\displaystyle\lim_{x\to2}\left(\frac{x+3}{x^2+x-6}\right)\]\\
	\vspace*{\fill}
	\vspace*{\fill}
	\vspace*{\fill}
	\vspace*{\fill}
	\crono
	\resetcrono{\beamerbutton{reset}}
\end{frame}

\begin{frame}
	\frametitle{Exercise 3}
	\vspace*{\fill}
	\vspace*{\fill}
	\vspace*{\fill}
	\initclock
	\Large
	\[\displaystyle\lim_{x\to2}\left(\frac{x+3}{x^2+x-6}\right)\]\\
	\vspace*{\fill}
	\vspace*{\fill}
	\vspace*{\fill}
	\vspace*{\fill}
	\crono
	\resetcrono{\beamerbutton{reset}}
\end{frame}

\begin{frame}
	\frametitle{Exercise 4}
	\vspace*{\fill}
	\vspace*{\fill}
	\vspace*{\fill}
	\initclock
	\Large
	\[\displaystyle\lim_{x\to2}\left(\frac{x+3}{x^2+x-6}\right)\]\\
	\vspace*{\fill}
	\vspace*{\fill}
	\vspace*{\fill}
	\vspace*{\fill}
	\crono
	\resetcrono{\beamerbutton{reset}}
\end{frame}

\begin{frame}
	\frametitle{Exercise 5}
	\vspace*{\fill}
	\vspace*{\fill}
	\vspace*{\fill}
	\initclock
	\Large
	\[\displaystyle\lim_{x\to2}\left(\frac{x+3}{x^2+x-6}\right)\]\\
	\vspace*{\fill}
	\vspace*{\fill}
	\vspace*{\fill}
	\vspace*{\fill}
	\crono
	\resetcrono{\beamerbutton{reset}}
\end{frame}
\end{document}