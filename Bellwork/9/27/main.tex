\documentclass[12pt]{beamer}
\usetheme{Warsaw}
\usepackage[utf8]{inputenc}
\usepackage{amsmath}
\usepackage{amsfonts}
\usepackage{amssymb}
\usepackage{graphicx}
\usepackage[font=Times,timeinterval=1,timeduration=2.0,timedeath=0,fillcolorwarningsecond=white!60!yellow,timewarningfirst=50,timewarningsecond=80,resetatpages=2]{tdclock}
\usepackage{tabularx}
\usepackage{array}
\usepackage{multicol}
\usepackage{longtable}
\usepackage{xcolor}
\usepackage{gensymb}
\usepackage{pgfplots}
\usepackage[makeroom]{cancel}

\graphicspath{ {./references/} }
\pgfplotsset{
	soldot/.style={color=black,only marks,mark=*},
	holdot/.style={color=black,fill=white,only marks,mark=*},
	compat=1.12
}
\newcolumntype{Y}{>{\centering\arraybackslash}X}
\makeatletter
\def\@listii{\leftmargin\leftmarginii
			  \topsep    2ex
			  \parsep    0\p@   \@plus\p@
			  \itemsep   \parsep}
\makeatother

\begin{document}
\begin{frame}
	\frametitle{Bellwork 9/27}
	\initclock

	\large
	\vfill
	\vfill
	Find $f'(3)$ if $f(x)=x^3$.\par
	\vfill
	Use $f'(a) = \displaystyle\lim_{x\to a}\left[\frac{f(x)-f(a)}{x-a}\right]$\par
	\vfill
	Recall: $z^3-y^3=(z-y)(z^2+zy+y^2)$\par
	\vfill
	\vfill
	\vfill

	\small
	\crono
	\resetcrono{\beamerbutton{reset}}
\end{frame}
\begin{frame}
	\frametitle{Bellwork 9/27 - Solution}

	\large
	\[f'(3) = \displaystyle\lim_{x\to 3}\left[\frac{f(x)-f(3)}{x-3}\right]\]
	\[=\displaystyle\lim_{x\to 3}\left(\frac{x^3-3^3}{x-3}\right)\]
	\[=\displaystyle\lim_{x\to 3}\left[\frac{\cancel{(x-3)}(x^2+3x+9)}{\cancel{x-3}}\right]\]
	\[=\displaystyle\lim_{x\to 3}\left(x^2+3x+9\right)=\boxed{27}\]
\end{frame}
\begin{frame}
	\frametitle{Exercise 1}

	\large
	\vfill
	\vfill
	\[f(x)=x^2+x\].\par

	\begin{enumerate}\itemsep1ex
		\item Find $f'(x)$\par
		\item Where does $f'(x)=0$?\par
		\item What is the equation of the tangent line at that point?\par
	\end{enumerate}
	\vfill
	\vfill
	\vfill
\end{frame}
\begin{frame}
	\frametitle{Exercise 1 - Solutions, Part 1}

	\large
	\[f'(x) = \displaystyle\lim_{h\to 0}\left[\frac{(x+h)^2+x+h-(x^2+x)}{h}\right]\]
	\[=\displaystyle\lim_{h\to 0}\left(\frac{x^2+2xh+h^2+x+h-x^2-x}{h}\right)\]
	\[=\displaystyle\lim_{h\to 0}\left(\frac{2xh+h^2+h}{h}\right)=\displaystyle\lim_{h\to 0}\left(2x+h+1\right)\]
	\[\implies\boxed{f'(x)=2x+1}\]
\end{frame}
\begin{frame}
	\frametitle{Exercise 1 - Solutions, Part 2}

	\[f'(x)=0\implies 2x+1=0\]
	\[\boxed{x=-\frac{1}{2}}\text{ or }\boxed{\text{at the point }\left(-\frac{1}{2}, -\frac{1}{4}\right)}\]
\end{frame}
\begin{frame}
	\frametitle{Exercise 1 - Solutions, Part 3}

	Equation of a line $g$:\par
	\begin{center}
		$g(x) = y = mx + b$ where $m$ is the slope and $b$ is the y-intercept.
	\end{center}
	\[f'(x)=0\implies m=0\]
	\[\implies y = b\]
	Since we have that $\left(-\frac{1}{2}, -\frac{1}{4}\right)$ lies on $f(x)$,\par
	\[b = -\frac{1}{4}\implies\boxed{y = -\frac{1}{4}}\text{ or }\boxed{g(x) = -\frac{1}{4}}\]
\end{frame}
\begin{frame}
	\frametitle{Exercise 2}

	\Large
	\begin{center}
		Find $f'(x)$ if $f(x) = \sqrt{2x+1}$
	\end{center}
\end{frame}
\begin{frame}
	\frametitle{Exercise 2 - Solution}

	\[f'(x) = \displaystyle\lim_{h\to 0}\left[\frac{\sqrt{2(x+h)+1}-\sqrt{2x+1}}{h}\right]\]
	\[=\displaystyle\lim_{h\to 0}\left[\frac{\sqrt{2(x+h)+1}-\sqrt{2x+1}}{h}\cdot\frac{\sqrt{2(x+h)+1}+\sqrt{2x+1}}{\sqrt{2(x+h)+1}+\sqrt{2x+1}}\right]\]
	\[=\displaystyle\lim_{h\to 0}\left\{\frac{2x+2h+1-2x-1}{h[\sqrt{2(x+h)+1}+\sqrt{2x+1}]}\right\}\]
	\[=\displaystyle \lim_{{h\to0}}\left[\frac{2\cancel{h}}{\cancel{h}(\sqrt{2(x+h)+1}+\sqrt{2x+1})}\right]=\boxed{\frac{1}{\sqrt{2x+1}}}\]
\end{frame}
\end{document}