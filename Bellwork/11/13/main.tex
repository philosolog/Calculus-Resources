\documentclass[12pt]{beamer}
\usetheme{Warsaw}
\usepackage[utf8]{inputenc}
\usepackage{amsmath}
\usepackage{amsfonts}
\usepackage{amssymb}
\usepackage{graphicx}
\usepackage[font=Times,timeinterval=1,timeduration=2.0,timedeath=0,fillcolorwarningsecond=white!60!yellow,timewarningfirst=50,timewarningsecond=80,resetatpages=2]{tdclock}
\usepackage{tabularx}
\usepackage{array}
\usepackage{multicol}
\usepackage{longtable}
\usepackage{xcolor}
\usepackage{textcomp, gensymb}
\usepackage{pgfplots}
\usepackage[makeroom]{cancel}

\graphicspath{ {./references/} }
\pgfplotsset{
	soldot/.style={color=black,only marks,mark=*},
	holdot/.style={color=black,fill=white,only marks,mark=*},
	compat=1.12
}
\newcolumntype{Y}{>{\centering\arraybackslash}X}
\makeatletter
\def\@listii{\leftmargin\leftmarginii
			  \topsep    2ex
			  \parsep    0\p@   \@plus\p@
			  \itemsep   \parsep}
\makeatother
\newcommand\at[2]{\left.#1\right|_{#2}}

\begin{document}
\begin{frame}
	\frametitle{Bellwork 11/13}
	\initclock

	\vfill
	\vfill
	\vfill
	\vfill
	\Large
	\[f(x)=\sin(x)\]
	\vfill
	\large
	\begin{center}
		Find the absolute extrema of $f$ for $x\in [0\text{, }\frac{3\pi}{2}]$.
	\end{center}
	\vfill
	\vfill
	\vfill
	\vfill

	\small
	\crono
	\resetcrono{\beamerbutton{reset}}
\end{frame}
\begin{frame}
	\frametitle{Bellwork 11/13 - Solution}

	Find where critical points occur, for $x\in (0\text{, }\frac{3\pi}{2})$:
	\[f'(x)=\cos(x)=0\]
	\[\implies x=\frac{\pi}{2}\]
	Test the candidates:
	\[x=0\text{, }\frac{\pi}{2}\text{, }\frac{3\pi}{2}\]
	\[f(0)=0\text{; }f\left(\frac{\pi}{2}\right)=1\text{; }f\left(\frac{3\pi}{2}\right)=-1\]
	\[\implies \boxed{\text{Absolute Maximum: }1\text{; }\text{Absolute Minimum: }-1}\]
\end{frame}
\begin{frame}
	\frametitle{Exercise 1}

	\vfill
	\vfill
	\vfill
	\LARGE
	\[g(x)=x^3-x^2\]
	\vfill
	\vfill
	\vfill
	\large
	Explain why $g$ satisfies the MVT on $x\in[0\text{, }1]$.\par
	\vfill
	Then, find the $x$-values that satisfy the conclusion of the MVT on this interval.
	\vfill
	\vfill
	\vfill
	\vfill
	\vfill
	\vfill
\end{frame}
\begin{frame}
	\frametitle{Exercise 1 - Solution}

	\large
	\[g'(a)=\frac{g(1)-g(0)}{1-0}\]
	\[3x^2-2x=0\]
	\[x^2=\frac{2}{3}x\]
	\[\boxed{x=0\text{, }\frac{2}{3}}\]
\end{frame}
\begin{frame}
	\frametitle{Exercise 2}

	\vfill
	\vfill
	\vfill
	\LARGE
	\[h(t)=2^t\]
	\vfill
	\vfill
	\vfill
	\large
	Explain why $h$ satisfies the MVT on $t\in[-2\text{, }2]$.\par
	\vfill
	Then, find the $t$-value that satisfies the conclusion of the MVT on this interval.
	\vfill
	\vfill
	\vfill
	\vfill
	\vfill
	\vfill
\end{frame}
\begin{frame}
	\frametitle{Exercise 2 - Solution}

	\large
	\[h'(a)=\frac{h(2)-h(-2)}{2-(-2)}\]
	\[\ln(2)2^t=\frac{2^2-2^{-2}}{4}\]
	\[\boxed{t=\log_{2}\left(\frac{\frac{2^2-2^{-2}}{4}}{\ln(2)}\right)}\]
\end{frame}
\end{document}