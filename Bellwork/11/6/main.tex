\documentclass[12pt]{beamer}
\usetheme{Warsaw}
\usepackage[utf8]{inputenc}
\usepackage{amsmath}
\usepackage{amsfonts}
\usepackage{amssymb}
\usepackage{graphicx}
\usepackage[font=Times,timeinterval=1,timeduration=2.0,timedeath=0,fillcolorwarningsecond=white!60!yellow,timewarningfirst=50,timewarningsecond=80,resetatpages=2]{tdclock}
\usepackage{tabularx}
\usepackage{array}
\usepackage{multicol}
\usepackage{longtable}
\usepackage{xcolor}
\usepackage{textcomp, gensymb}
\usepackage{pgfplots}
\usepackage[makeroom]{cancel}

\graphicspath{ {./references/} }
\pgfplotsset{
	soldot/.style={color=black,only marks,mark=*},
	holdot/.style={color=black,fill=white,only marks,mark=*},
	compat=1.12
}
\newcolumntype{Y}{>{\centering\arraybackslash}X}
\makeatletter
\def\@listii{\leftmargin\leftmarginii
			  \topsep    2ex
			  \parsep    0\p@   \@plus\p@
			  \itemsep   \parsep}
\makeatother
\newcommand\at[2]{\left.#1\right|_{#2}}

\begin{document}
\begin{frame}
	\frametitle{Bellwork 11/6}
	\initclock

	\vfill
	\vfill
	\vfill
	\Large
	A particle's motion can be described by the following elliptical equation: \[2x^2+xy+2y^2=8\]
	When the particle is at $(-1,0)$, $\frac{\mathrm{d}x}{\mathrm{d}t}=4$.
	\vfill
	Find $\frac{\mathrm{d}y}{\mathrm{d}t}$.
	\vfill
	\vfill
	\vfill

	\small
	\crono
	\resetcrono{\beamerbutton{reset}}
\end{frame}
\begin{frame}
	\frametitle{Bellwork 11/6 - Solution}

	Implicitly Differentiate:
	\[4x\left(\frac{\mathrm{d}x}{\mathrm{d}t}\right)+y\left(\frac{\mathrm{d}x}{\mathrm{d}t}\right)+x\left(\frac{\mathrm{d}y}{\mathrm{d}t}\right)+4y\left(\frac{\mathrm{d}y}{\mathrm{d}t}\right)=0\]
	\[\frac{\mathrm{d}y}{\mathrm{d}t}=\frac{-4x-y}{x+4y}\left(\frac{\mathrm{d}x}{\mathrm{d}t}\right)\]
	Substitute:
	\[\at{\frac{\mathrm{d}x}{\mathrm{d}t}}{-1}=\left(\frac{4}{-1}\right)\left(4\right)\]
	\[\boxed{\at{\frac{\mathrm{d}x}{\mathrm{d}t}}{-1}=-16}\]
\end{frame}
\begin{frame}
	\frametitle{Exercise 1}

	Find the linear approximation of the function $f(x)=\sqrt{1+x}$ at $x=3$ and use it to approximate the numbers $\sqrt{3}$ and $\sqrt{5}$.
\end{frame}
\begin{frame}
	\frametitle{Exercise 1 - Solution, Part 1}

	\large
	\[f(x)=\sqrt{1+x}\implies \sqrt{3}=\sqrt{1+2}\text{ and }\sqrt{5}=\sqrt{1+4}\]
	\[f'(x)=\frac{1}{2\sqrt{1+x}}\]
	\vfill
	Use the equation of the line tangent to $f$ at $x=3$:
	\[g(x)=f'(3)(x-3)+f(3)\]
	\[g(x)=\frac{1}{4}(x-3)+2\]
\end{frame}
\begin{frame}
	\frametitle{Exercise 1 - Solution, Part 2}

	\large
	Since $f(x)\approx g(x)$, we find $g(2)$ and $g(4)$:
	\[f(2)\approx g(2)\implies f(2)\approx \frac{1}{4}(-1)+2\implies \boxed{f(2)\approx 1.75}\]
	\[f(4)\approx g(4)\implies f(4)\approx \frac{1}{4}(1)+2\implies \boxed{f(4)\approx 2.25}\]
\end{frame}
\begin{frame}
	\frametitle{Exercise 2}

	\Large
	\begin{center}
		Estimate $e^{0.75}$ using linear approximation.
	\end{center}
\end{frame}
\begin{frame}
	\frametitle{Exercise 2 - Solution}

	\[f(x)=e^{1-x^2}\implies e^{0.75}=e^{1-0.5^2}=f(0.5)\]
	\[f'(x)=-2e^{1-x^2}x\]
	\vfill
	Use the equation of the line tangent to $f$ at $x=1$:
	\[g(x)=f'(1)(x-1)+f(1)\]
	\[g(x)=-2(x-1)+1\]
	Since $f(0.5)\approx g(0.5)$:
	\[f(0.5)\approx -2(0.5-1)+1\]
	\[\boxed{f(0.5)\approx 2}\]
\end{frame}
\end{document}