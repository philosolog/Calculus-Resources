\documentclass[12pt]{beamer}
\usetheme{Warsaw}
\usepackage[utf8]{inputenc}
\usepackage{amsmath}
\usepackage{amsfonts}
\usepackage{amssymb}
\usepackage{graphicx}
\usepackage[font=Times,timeinterval=1,timeduration=2.0,timedeath=0,fillcolorwarningsecond=white!60!yellow,timewarningfirst=50,timewarningsecond=80,resetatpages=2]{tdclock}
\usepackage{tabularx}
\usepackage{array}
\usepackage{multicol}
\usepackage{longtable}
\usepackage{xcolor}
\usepackage{textcomp, gensymb}
\usepackage{pgfplots}
\usepackage[makeroom]{cancel}

\graphicspath{ {./references/} }
\pgfplotsset{
	soldot/.style={color=black,only marks,mark=*},
	holdot/.style={color=black,fill=white,only marks,mark=*},
	compat=1.12
}
\newcolumntype{Y}{>{\centering\arraybackslash}X}
\makeatletter
\def\@listii{\leftmargin\leftmarginii
			  \topsep    2ex
			  \parsep    0\p@   \@plus\p@
			  \itemsep   \parsep}
\makeatother
\newcommand\at[2]{\left.#1\right|_{#2}}

\begin{document}
\begin{frame}
	\frametitle{Bellwork 11/3}
	\initclock

	\vfill
	\vfill
	\vfill
	\vfill
	\Large
	Suppose $y=3^{x\ln(x)}$, where $x$ and $y$ are functions of $t$.
	\vfill
	\vfill
	If $\frac{\mathrm{d}x}{\mathrm{d}t} = \frac{1}{\ln(3)}$, find $\frac{\mathrm{d}y}{\mathrm{d}t}$ when $x=1$.
	\vfill
	\vfill
	\vfill
	\vfill
	\vfill
	\vfill
	\vfill

	\small
	\crono
	\resetcrono{\beamerbutton{reset}}
\end{frame}
\begin{frame}
	\frametitle{Bellwork 11/3 - Solution, Part 1}

	\large
	Logarithmic Differentiation:
	\[\ln(y)=x\ln(x)\ln(3)\]
	\[\frac{1}{y}\left(\frac{\mathrm{d}y}{\mathrm{d}t}\right)=\ln(3)[\ln(x)+1]\left(\frac{\mathrm{d}x}{\mathrm{d}t}\right)\]
	\[\frac{\mathrm{d}y}{\mathrm{d}t}=y\ln(3)[\ln(x)+1]\left(\frac{\mathrm{d}x}{\mathrm{d}t}\right)\]
	\[\frac{\mathrm{d}y}{\mathrm{d}t}=\ln(3)\left[3^{x\ln(x)}\right][\ln(x)+1]\left(\frac{\mathrm{d}x}{\mathrm{d}t}\right)\]
\end{frame}
\begin{frame}
	\frametitle{Bellwork 11/3 - Solution, Part 2}

	\large
	Substitute:
	\[\frac{\mathrm{d}y}{\mathrm{d}t}=\ln(3)\left[3^{(1)\ln(1)}\right][\ln(1)+1]\left[\frac{1}{\ln(3)}\right]\]
	\[\boxed{\frac{\mathrm{d}y}{\mathrm{d}t}=1}\]
\end{frame}
\begin{frame}
	\frametitle{Exercise 1}

	\vfill
	\vfill
	\vfill
	\Large
	A particle's motion can be described by the following elliptical equation: \[x^2+xy+y^2=4\]
	When the particle is at $(-2,0)$, $\frac{\mathrm{d}x}{\mathrm{d}t}=-3$.
	\vfill
	Find $\frac{\mathrm{d}y}{\mathrm{d}t}$.
	\vfill
	\vfill
	\vfill
\end{frame}
\begin{frame}
	\frametitle{Exercise 1 - Solution}

	\large
	Implicitly Differentiate:
	\[2x\left(\frac{\mathrm{d}x}{\mathrm{d}t}\right)+y\left(\frac{\mathrm{d}x}{\mathrm{d}t}\right)+x\left(\frac{\mathrm{d}y}{\mathrm{d}t}\right)+2y\left(\frac{\mathrm{d}y}{\mathrm{d}t}\right)=0\]
	\[\frac{\mathrm{d}y}{\mathrm{d}t}=\frac{-2x-y}{x+2y}\left(\frac{\mathrm{d}x}{\mathrm{d}t}\right)\]
	Substitute:
	\[\frac{\mathrm{d}y}{\mathrm{d}t}=\frac{4}{-2}(-3)\]
	\[\boxed{\frac{\mathrm{d}y}{\mathrm{d}t}=6}\]
\end{frame}
\end{document}