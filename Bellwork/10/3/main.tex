\documentclass[12pt]{beamer}
\usetheme{Warsaw}
\usepackage[utf8]{inputenc}
\usepackage{amsmath}
\usepackage{amsfonts}
\usepackage{amssymb}
\usepackage{graphicx}
\usepackage[font=Times,timeinterval=1,timeduration=2.0,timedeath=0,fillcolorwarningsecond=white!60!yellow,timewarningfirst=50,timewarningsecond=80,resetatpages=2]{tdclock}
\usepackage{tabularx}
\usepackage{array}
\usepackage{multicol}
\usepackage{longtable}
\usepackage{xcolor}
\usepackage{textcomp, gensymb}
\usepackage{pgfplots}
\usepackage[makeroom]{cancel}

\graphicspath{ {./references/} }
\pgfplotsset{
	soldot/.style={color=black,only marks,mark=*},
	holdot/.style={color=black,fill=white,only marks,mark=*},
	compat=1.12
}
\newcolumntype{Y}{>{\centering\arraybackslash}X}
\makeatletter
\def\@listii{\leftmargin\leftmarginii
			  \topsep    2ex
			  \parsep    0\p@   \@plus\p@
			  \itemsep   \parsep}
\makeatother

\begin{document}
\begin{frame}
	\frametitle{Bellwork 10/3}
	\initclock

	\large
	\vfill
	\vfill
	A particle moves along the $x$-axis. Its position can be described by the equation: \[x(t)=t^2+t\]
	Find $a(t)$ or $x''(t)$, the function that describes the acceleration of this particle.\par
	\vfill
	\[\text{Recall: } x'(t)=\displaystyle\lim_{h\to 0}\left[\frac{x(t+h)-x(t)}{h}\right]\]
	\vfill
	\vfill

	\small
	\crono
	\resetcrono{\beamerbutton{reset}}
\end{frame}
\begin{frame}
	\frametitle{Bellwork 10/3 - Solution}

	\large
	\[v(t)=\displaystyle\lim_{h\to 0}\left[\frac{(t+h)^2+(t+h)-(x^2+x)}{h}\right]=2t+1\]
	\[\implies a(t)=v'(t)=\displaystyle\lim_{h\to 0}\left[\frac{2(t+h)+1-(2t+1)}{h}\right]\]
	\[\implies \boxed{a(t)=2}\]
\end{frame}
\begin{frame}
	\frametitle{Exercise 1}

	\large
	A particle moves along the $x$-axis. Its position can be modeled by: \[x(t)=t^3-3t^2-9t-1\]
	\vfill
	\Large
	\begin{enumerate}\itemsep2ex
		\item When is the particle moving to the right?
		\item When is the particle moving to the left?
	\end{enumerate}
\end{frame}
\begin{frame}
	\frametitle{Exercise 1 - Solutions}

	First, we find where the particle changes directions:\par
	\[v(t)=3t^2-6t-9=0\]
	\[\implies 3(t-2t-3)=0\]
	\[\implies t=-1,3\]
	\vfill
	Next, we test $t$ in $(-\infty, -1)$, $(-1, 3)$, and $(3, \infty)$ to get:\par
	\vfill
	\small
	\begin{enumerate}
		\item The particle is $\boxed{\text{moving to the right when }t<-1\text{ or }t>3}$.
		\item The particle is $\boxed{\text{moving to the left when }-1<t<3}$.
	\end{enumerate}
\end{frame}
\begin{frame}
	\frametitle{Exercise 2}

	From Exercise 1:\par
	\[x(t)=t^3-3t^2-9t-1\]
	\[v(t)=3t^2-6t-9\]
	The particle is moving to the right when $t<-1$ or $t>3$.\par
	The particle is moving to the left when $-1<t<3$.\par
	\vfill
	\vfill
	\large
	When is the particle..\par
	\vfill
	\begin{enumerate}\itemsep2ex
		\item Speeding up?
		\item Slowing down?
	\end{enumerate}
	\vfill
\end{frame}
\begin{frame}
	\frametitle{Exercise 2 - Solutions, Part 1}

	\large
	The particle is speeding up when its velocity and acceleration have the same sign. It is slowing down when the particle's velocity and acceleration have opposite signs.\par
\end{frame}
\begin{frame}
	\frametitle{Exercise 2 - Solutions, Part 2}

	We have found:\par
	\begin{center}
		$v(t)>1$ if $t<-1$ or $t>3$, and $v(t)<1$ if $-1<t<3$
	\end{center}
	\vfill
	Now, we must find where $a(t)$ changes signs:\par
	\[a(t)=6t-6=0\]
	\[\implies 6(t-1)=0\implies t=1\]
	Then, we test values for $t<1$ and $t>1$, so:\par
	\begin{center}
		$a(t)>1$ if $t>1$, and $a(t)<1$ if $t<1$
	\end{center}
	The particle $\boxed{\text{is speeding up when }-1<t<1\text{ or }t>3}$ and $\boxed{\text{slowing down when }t<-1\text{ or }1<t<3}$.
\end{frame}
\begin{frame}
	\frametitle{Exercise 3}

	\large
	Another particle moves along the $x$-axis. Its position can be described by the following: \[x(t)=e^t-t\]
	\vfill
	When is the particle moving to the right? Left?
	\vfill
	\vfill
\end{frame}
\begin{frame}
	\frametitle{Exercise 3 - Solutions}

	Find $t$ where $v(t)=0$:\par
	\[v(t)=e^t-1=0\]
	\[\implies e^t=1\]
	\[\implies t=0\]
	\vfill
	Now, we find the sign of $v(t)$ for $t<0$ and $t>0$:
	\[v(-1)=e^{-1}-1<0\]
	\[v(1)=e^{1}-1>0\]
	$\therefore$ the particle moves $\boxed{\text{to the right when }t>0}$, and $\boxed{\text{to the left when }t<0}$.
\end{frame}
\end{document}