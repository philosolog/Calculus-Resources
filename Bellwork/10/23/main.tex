\documentclass[12pt]{beamer}
\usetheme{Warsaw}
\usepackage[utf8]{inputenc}
\usepackage{amsmath}
\usepackage{amsfonts}
\usepackage{amssymb}
\usepackage{graphicx}
\usepackage[font=Times,timeinterval=1,timeduration=2.0,timedeath=0,fillcolorwarningsecond=white!60!yellow,timewarningfirst=50,timewarningsecond=80,resetatpages=2]{tdclock}
\usepackage{tabularx}
\usepackage{array}
\usepackage{multicol}
\usepackage{longtable}
\usepackage{xcolor}
\usepackage{textcomp, gensymb}
\usepackage{pgfplots}
\usepackage[makeroom]{cancel}

\graphicspath{ {./references/} }
\pgfplotsset{
	soldot/.style={color=black,only marks,mark=*},
	holdot/.style={color=black,fill=white,only marks,mark=*},
	compat=1.12
}
\newcolumntype{Y}{>{\centering\arraybackslash}X}
\makeatletter
\def\@listii{\leftmargin\leftmarginii
			  \topsep    2ex
			  \parsep    0\p@   \@plus\p@
			  \itemsep   \parsep}
\makeatother

\begin{document}
\begin{frame}
	\frametitle{Bellwork 10/23}
	\initclock

	
\end{frame}
\begin{frame}
	\frametitle{Bellwork 10/23 - Solution}

	
\end{frame}
\begin{frame}
	\frametitle{Exercise 1}

	\begin{tabular}{|c||c|c|} \hline
		$x$ & 1 & 3 \\ \hline
		$f(x)$ & $-\frac\pi2$ & $\frac\pi2$ \\ \hline
		$f'(x)$ & 0 & $-1$ \\ \hline
		$f''(x)$ & 3 & 2 \\ \hline
	\end{tabular}\par
	$g(x)=$
\end{frame}
\begin{frame}
	\frametitle{Exercise 1 - Solution}

	
\end{frame}
\begin{frame}
	\frametitle{Exercise 2}

	
\end{frame}
\begin{frame}
	\frametitle{Exercise 2 - Solution}

	
\end{frame}
\begin{frame}
	\frametitle{Exercise 3}

	
\end{frame}
\begin{frame}
	\frametitle{Exercise 3 - Solution}

	
\end{frame}
\end{document}