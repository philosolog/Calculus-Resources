\documentclass[12pt]{beamer}
\usetheme{Warsaw}
\usepackage[utf8]{inputenc}
\usepackage{amsmath}
\usepackage{amsfonts}
\usepackage{amssymb}
\usepackage{graphicx}
\usepackage[font=Times,timeinterval=1,timeduration=2.0,timedeath=0,fillcolorwarningsecond=white!60!yellow,timewarningfirst=50,timewarningsecond=80,resetatpages=2]{tdclock}
\usepackage{tabularx}
\usepackage{array}
\usepackage{multicol}
\usepackage{longtable}
\usepackage{xcolor}

\newcolumntype{Y}{>{\centering\arraybackslash}X}
\begin{document}
	\begin{frame}
		\frametitle{Bellwork 8/14 (5 Minutes)}
		\initclock
		\Large
		\textbf{[Domains]}\vspace{.2cm}\\
		\vspace*{\fill}
		Find the domain:
		\vspace*{\fill}
		\[\frac{\alpha-3}{\alpha^2-6\alpha+8}\leq 0\]\\
		\vspace*{\fill}
		\vspace*{\fill}
		\vspace*{\fill}
		\vspace*{\fill}
		\crono
		\resetcrono{\beamerbutton{reset}}
	\end{frame}
	\begin{frame} % Equation of a parallel line that passes through a point.
		\frametitle{Exercise}
		\initclock
		\paragraph*{question} Points (1,1) and (3, 8) are on a line. Find an equation for the line that is parallel to this line and passes through (-1, 4).
		\crono
		\resetcrono{\beamerbutton{reset}}
	\end{frame}
\end{document}